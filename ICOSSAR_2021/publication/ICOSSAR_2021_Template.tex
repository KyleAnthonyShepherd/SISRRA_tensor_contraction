%%%%%%%%%%%%%%%%%%%%%%%%%%%%%%%%%%%%%%%%%%%%%%%%%%%%%%%%%%%%%%%%%%%%%%%%%%%
%%  ICOSSAR2021-Full-paper.tex:   24/09/2020          			         %%
%%  The content, structure, format and layout of this style file is the  %%
%%  property of Research Publishing Services                             %%
%%  Copyright (c) 2020 Research Publishing Services,                     %%
%%  All rights are reserved.                                             %%
%%%%%%%%%%%%%%%%%%%%%%%%%%%%%%%%%%%%%%%%%%%%%%%%%%%%%%%%%%%%%%%%%%%%%%%%%%%

\documentclass[12pt,twocolumn]{article}
\setlength{\columnsep}{7mm}

% All packages needed are put in 'Settings.tex' file
% -------------- Packages
% basic packages and settings - please don't change it
\usepackage{xfrac}
\usepackage{abstract}
\renewcommand{\abstractname}{}    % clear the title
\renewcommand{\absnamepos}{empty} % originally center
\usepackage{float}
\usepackage[utf8]{inputenc}
\usepackage{fontspec}
\usepackage[T1]{fontenc}
\usepackage{mathptmx}
\usepackage[fleqn]{amsmath}
\setlength{\mathindent}{0pt}
\usepackage{geometry}
\geometry{
 a4paper,
 total={160mm,247mm},
 left=25mm,
 right=25mm,
 top=25mm,
 bottom=25mm
}
\usepackage{mwe}
\usepackage{afterpage}

\usepackage{graphicx}
\usepackage[export]{adjustbox}
\usepackage[compatV3]{fancyhdr}
\fancyhf{}

\usepackage{caption}
\captionsetup[table]{skip=-10pt, font=footnotesize, labelformat=simple}
\captionsetup[figure]{skip=10pt,font=footnotesize, labelformat=simple}

\usepackage{enumerate}
\usepackage{hyperref}
\hypersetup{hidelinks}

\usepackage{authblk}
\renewcommand*{\Authfont}{\twelve}
\renewcommand*{\Affilfont}{\nine\it\vspace*{-3pt}}
\setlength{\affilsep}{12pt}

% If no hyphen is allowed
% \usepackage[none]{hyphenat}

\usepackage{color}

% For the header font type
% Arial
\setsansfont[
ItalicFont=ariali.ttf,
]{arial.ttf}

\usepackage{fancyhdr}
% customerized fancyhdr
\pagestyle{fancy}
\setlength{\voffset}{-3mm}
\fancyhead{}
\fancyfoot{}
\fancyhead[L]{\includegraphics[height=25mm]{figures/IASSAR_Logo.png}}
\fancyhead[R]{
\includegraphics[height=18mm]{figures/ICOSSAR2021_Logo.png}\\
\vspace*{-10pt}
\ten
\color{blue}
{
\sffamily
\textit{
The 13th International Conference on Structural\\
Safety and Reliability (ICOSSAR 2021),\\
June 21-25, 2021, Shanghai, P.R. China\\
J. Li, Pol D. Spanos, J.B. Chen \& Y.B. Peng (Eds)}
}
}
\renewcommand{\headrulewidth}{0pt}
\setlength{\headsep}{0pt}
% \setlength{\headheight}{103.56499pt}
% user required packages or settings


\usepackage{booktabs}
% % \usepackage{capt-of}
% % \usepackage{ragged2e}
%
\usepackage{tabularx}
%
% \usepackage{array}
% \newcolumntype{L}[1]{>{\raggedright\let\newline\\\arraybackslash\hspace{0pt}}m{#1}}
% \newcolumntype{C}[1]{>{\centering\let\newline\\\arraybackslash\hspace{0pt}}m{#1}}
% \newcolumntype{R}[1]{>{\raggedleft\let\newline\\\arraybackslash\hspace{0pt}}m{#1}}
%
% \renewcommand{\arraystretch}{1.3}
% \renewcommand{\floatpagefraction}{1}
% \renewcommand{\textfraction}{0}
% \renewcommand{\dbltopfraction }{1}
% \renewcommand{\topfraction }{1}

% -------------- FontSize --------------
\newcommand{\seven}{\fontsize{7pt}{\baselineskip}\selectfont}    % 7pt
\newcommand{\nine}{\fontsize{9pt}{\baselineskip}\selectfont}    % 9pt
\newcommand{\ten}{\fontsize{10pt}{\baselineskip}\selectfont}    % 10pt
\newcommand{\twelve}{\fontsize{12pt}{\baselineskip}\selectfont} % 12pt
\newcommand{\fourt}{\fontsize{14pt}{\baselineskip}\selectfont}  % 14pt
\newcommand{\sixt}{\fontsize{16pt}{\baselineskip}\selectfont}   % 16pt



\usepackage{titlesec}
% Fontsize of sections
\titleformat{\section}
	{\fontsize{12}{0}\selectfont}{\thesection}{1em}{\normalfont}
% Fontsize of subsections
\titleformat{\subsection}
	{\fontsize{12}{0}\selectfont}{\thesubsection}{1em}{\itshape}
% Fontsize of subsubsections
\titleformat{\subsubsection}
	{\fontsize{12}{0}\selectfont}{\thesubsubsection}{1em}{\itshape}

\usepackage{etoolbox}

\usepackage{academicons}
\usepackage{xcolor}

\newcommand{\orcid}[1]{\href{https://orcid.org/#1}{\textcolor[HTML]{A6CE39}{\aiOrcid}}}


% All packages needed are put in 'Settings.tex' file

\begin{document}

% Font size in tables
\BeforeBeginEnvironment{tabular}{\begin{center}\ten}
\AfterEndEnvironment{tabular}{\end{center}}

% Here put the title name ...
\newcommand{\mytitle}{Tensor Network Contraction For Network Reliability Estimates}

% Please DO NOT change it...
\title{\sixt \bf \vspace*{25mm} \mytitle \vspace*{-20pt}}

% Here put the author name(s)...
\author[a]{Kyle Shepherd \orcid{0000-0002-8220-7448}}
\author[b]{Leonardo Dueñas-Osorio \orcid{0000-0002-7138-7746}}

% Here put the institute name(s) and e-mail(s)...
\affil[a]{PhD Student, Dept. of Civil Engineering, Rice University, Texas, Houston, United States, E-mail: kas20@rice.edu, ORCID:0000-0002-8220-7448}
\affil[b]{Professor, Dept. of Civil Engineering, Rice University, Houston, United States, E-mail: leonardo.duenas-osorio@rice.edu, ORCID:0000-0002-7138-7746}

% Please DO NOT change it...
\pagenumbering{gobble}
% >>>>>>>>>>>>>>>>>>>>>>>>>>>>>>>>>>>>>>>>>>>>>>>>>>>>>
\twocolumn[

\date{}
\maketitle
\thispagestyle{fancy}
\vspace*{-36pt}

% Here start your abstract...
\begin{abstract}
	\noindent

	ABSTRACT: Quantifying network reliability is a hard problem, proven to be \#P-complete \cite{valiant1979complexity}. For real-world network planning and decision making, approximations for the network reliability problem are necessary. This study shows that tensor network contraction (TNC) methods can quickly estimate an upper bound of All Terminal Reliability, $Rel_{ATR}(G)$, by solving a superset of the network reliability problem: the edge cover problem, $EC(G)$. In addition, these tensor contraction methods can exactly solve source-terminal (S-T) reliability for the class of directed acyclic networks, $Rel_{S-T}(G)$.

    The computational complexity of TNC methods is parameterized by treewidth, significantly benefitting from recent advancements in approximate tree decomposition algorithms \cite{dell2018pace}. This parameterization does not rely on the reliability of the graph, which means these tensor contraction methods can determine reliability faster than Monte Carlo methods on highly reliable networks, while also providing exact answers or guaranteed upper bound estimates. These tensor contraction methods are applied to grid graphs, random cubic graphs, and a selection of 58 power transmission networks \cite{li2016characterizing}, demonstrating computational efficiency and effective approximation using $EC(G)$.
    %
    % Most network reliability estimations are performed using randomized methods that output probabilistic bounds, such as direct Monte Carlo simulation (MCS) or fully polynomial randomized approximation schemes (FPRAS) \cite{karger2001randomized}. In contrast, our method exactly bounds the uncertainty of the true value, so as to better inform stakeholders of the worst case scenarios in their critical infrastructure missions.
    %
    % Exact bounds are obtained in this study by solving a superset of the network reliability problem: the edge cover problem. Despite being \#P-complete, many practical solvers exist such as d4 \cite{lagniez2017improved}. However, the topological structures of infrastructure networks can be exploited anew by endowing nodes and links of the networks with tensors, whose logic maps onto the constraints of counting edge covers. The resulting tensor contractions take advantage of the structure of common infrastructure networks and solve many cases that existing competitive solvers cannot solve.
    %
    % We show that our tensor contractions offer computational advantages relative to state-of-the-art methods on problems with manageable treewidth (which can now be efficiently approximated \cite{dell2018pace}), and with favorable topological properties consistent with engineered networks, such as low degree nodes. Our method can also exploit progress on linear algebra and highly optimized matrix operation libraries on CPUs and GPUs, unlike alternatives that rely on conditional operations.
    %
    % In sum, tensor contractions can practically bound network reliability estimates for decision making, and strike a balance between asymptotic errors and guaranteed error estimation for critical infrastructure applications.
\end{abstract}


\vspace*{20pt}
% Please keep this closing bracket to complete the single column format for abstract.
]

% >>>>>>>>>>>>>>>>>>>>>>>>>>>>>>>>>>>>>>>>>>>>>>>>>>>>>

% Here start your contexts...

\BeforeBeginEnvironment{tabularx}{\begin{center}\seven}
\AfterEndEnvironment{tabularx}{\end{center}}

\addtolength{\textfloatsep}{-0.25in}
\input{manuscript.tex}
% Here start your contexts...

\newpage
\clearpage
\section{REFERENCES}
{\small
\bibliographystyle{unsrt}
\bibliography{W:/Kyle_Shepherd/bib/bib}
}
\end{document}
